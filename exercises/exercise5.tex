\documentclass[./../main.tex]{subfiles}
\graphicspath{{img/}}
\begin{document}
	\begin{exercise}
		¿Cómo funciona y qué mide un calorímetro (en física de partículas)? ¿De qué materiales se pueden construir?

		\begin{solution}
			Un calorímetro es un dispositivo que absorbe la energía cinética total de una partícula y emite una señal proporcional a la energía depositada. Es decir, mide la energía y la dirección de las partículas.

			Los calorímetros se pueden construir de detectores de materiales homogéneos, por ejemplo un centellador. También pueden estar compuestos de varias de capas absorbentes (\emph{e.g.}, un metal como el plomo) o bien pueden ser detectores (centelladores, MWPC, etc.). Estos últimos se conocen como detectores de \emph{prueba} (\emph{sampling detectors}, por su nombre en inglés), que durante el proceso de absorción de partículas, se generan partículas secundarias, lo cual generará aún más partículas y es por eso que los calorímetros también se conocen como detectores de cascada.
		\end{solution}
	\end{exercise}
\end{document}
