\documentclass[./../main.tex]{subfiles}
\graphicspath{{img/}}
\begin{document}
	\begin{exercise}
		¿Cuáles son los ángulos de Cherenkov para electrones y piones con momento de \qty{1000}{\MeV\per\clight} para un radiador con índice de refracción \(n = 1.4\)?

		\begin{solution}
			Los ángulos de Cherenkov están dados por

			\begin{equation*}
				\cos\theta_{C} = \dfrac{1}{\beta n},
			\end{equation*}

			donde \(\beta\) es el factor de la velocidad relativa y \(n\) es el índice de refracción.

			Para obtener \(\beta\) fácilmente, primero obtenemos el factor \(\gamma\) dado como

			\begin{equation*}
				\gamma = \dfrac{pc}{E} + 1,
			\end{equation*}

			con \(E = mc^{2}\).

			Por un lado, para el electrón, cuya masa es \(m_{\e} = \qty{0.511}{\MeV\per\clightsq}\),

			\begin{align*}
				\gamma_{\e} &= \dfrac{(\qty{1000}{\MeV\per\clight}\cdot c)}{(\qty{0.511}{\MeV\per\clightsq})\cdot c^{2}} + 1,\\
				&= \dfrac{\qty{1000}{\MeV\per\clight}}{\qty{0.511}{\MeV\per\clightsq}} + 1,\\
				\Aboxedsec{\gamma_{\e} &= \num{1957.95}.}
			\end{align*}

			Y el factor \(\beta_{\e}\),

			\begin{align*}
				\beta_{\e} &= \sqrt{1 - \dfrac{1}{(\num{1957.95})^{2}}},\\
				\Aboxedsec{\beta_{\e} &= \num{0.9999}.}
			\end{align*}

			Por lo que los ángulos de Cherenkov para los electrones son

			\begin{align*}
				\theta_{C} &= \arccos(\tfrac{1}{\beta_{\e} n}),\\
				&= \arccos\left(\dfrac{1}{(\num{0.9999})(1.4)}\right),\\
				\Aboxedmain{\theta_{C} &= \qty{44.4153}{\degree}.}
			\end{align*}

			Por el otro, para el pión, cuya masa es \(m_{\pi^{+}} = \qty{140}{\MeV\per\clightsq}\),

			\begin{align*}
				\gamma_{\pi^{+}} &= \dfrac{(\qty{1000}{\MeV\per\clight})\cdot c}{(\qty{140}{\MeV\per\clightsq})\cdot c^{2}} + 1,\\
				&= \dfrac{\qty{1000}{\MeV}}{\qty{140}{\MeV}} + 1,\\
				\Aboxedsec{\gamma_{\pi^{+}} &= \num{8.14286}.}
			\end{align*}

			Y el factor \(\beta_{\pi^{+}}\),
			

			\begin{align*}
				\beta_{\pi^{+}} &= \sqrt{1 - \dfrac{1}{(\num{8.14286})^{2}}},\\
				\Aboxedsec{\beta_{\pi^{+}} &= \num{0.992431}.}
			\end{align*}

			Por lo que los ángulos de Cherenkov para los piones son

			\begin{align*}
				\theta_{C} &= \arccos(\tfrac{1}{\beta_{\pi^{+}} n}),\\
				&= \arccos\left(\dfrac{1}{(\num{0.992431})(1.4)}\right),\\
				\Aboxedmain{\theta_{C} &= \qty{43.9675}{\degree}.}
			\end{align*}
		\end{solution}
	\end{exercise}
\end{document}
